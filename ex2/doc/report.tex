%!TEX TS-program = xelatex
%!TEX encoding = UTF-8 Unicode
\documentclass[11pt,a4paper]{article}

%\usepackage[left=70pt,top=50pt,bottom=70pt,right=40pt]{geometry}
\usepackage{amsmath}
\usepackage{amsfonts}
\usepackage{amssymb}
\usepackage{fixltx2e}
\usepackage{cmap}
\usepackage{enumerate}
\usepackage{ifthen}
\usepackage{listings}
\usepackage{url}
\usepackage[T1]{fontenc}
%\usepackage{fontspec}
%\usepackage{xunicode}
%\usepackage{xltxtra}
%\setmainfont[Mapping=tex-text,Ligatures={Common,Rare,Discretionary}]{Linux Libertine O}
\usepackage{pdflscape}
\usepackage{alltt}
%\usepackage{algpseudocode}
%\usepackage{wrapfig}
%\usepackage{graphicx}

\ifthenelse{\isundefined{\hypersetup}}{
    \usepackage[colorlinks=true,linkcolor=blue,urlcolor=blue]{hyperref}
    \urlstyle{same}
}{}

\hypersetup{
    pdftitle={Intelligent Agents - EX2 - Yoan Blanc, Tiziano Signo}
}
\title{\phantomsection%
    A Deliberative Agent for the Pickup and Delivery Problem
}
\author{
    \textbf{Group 16}\\
    Yoan Blanc \texttt{<yoan.blanc@epfl.ch>}, 213552\\
    Tiziano Signo \texttt{<tiziano.signo@epfl.ch>}, 226511
}
\date{\today}


\begin{document}
\maketitle

\noindent
\begin{quote}{\it

    In this exercise, you will learn to use a deliberative agent to solve the
    Pickup and Delivery Problem. A deliberative agent does not simply react to
    percepts coming from the environment. It can build a plan that specifies
    the sequence of actions to be taken in order to reach a certain goal. A
    deliberative agent has goals (e.g. to deliver all tasks) and is fully aware
    of the world it is acting in.

    Unlike the reactive agent, the deliberative agent knows the list of tasks
    that must be delivered. The deliberative agent can therefore construct a
    plan (a certain path through the network) that guarantees the optimal
    delivery of tasks.

    \begin{enumerate}
        \item Choose a representation for the states, transitions and goals
            (final states) to be used in a state-based search algorithm that
            finds the optimal plan for delivering a set of tasks.

        \item Implement the state-based \emph{breadth-first search} and
            \emph{A* heuristic search} algorithms.  Choose one heuristic and
            explain why. Discuss the optimality of your new algorithm in
            relation to your heuristic.

        \item Implement a deliberative agent which can use the above planning
            algorithms.

        \item Compare the performances of the \emph{breadth-first search} and
            the \emph{A* search} algorithms for different problem sizes.

        \item Run the simulation with $1$, $2$ and $3$ deliberative agents and
            report the differences of the joint performance of the agents.

    \end{enumerate}

}\end{quote}

\newpage
\medskip
\textbf{State representation}

The objects of the system are the tasks, cities.

$$ objects = \{T_1, \cdots, T_n\} \cup \{city_1, \cdots, city_m\} $$

Six predicates are used to define where the agent is, how it is doing and in
which state are the tasks.

\begin{align*}
predicates = &city(c)                            & c \text{is a city} \\
             &task(t)                            & t \text{is a task} \\
             &position(c)                        & \text{position of the agent}      \\
             &capacity(n)                        & \text{capacity of the agent}      \\
             &ready(t, from, to, weight, reward) & \text{task ready to be picked-up} \\
             &loaded(t, to, weight, reward)      & \text{task currently in transit}  \\
             &balance(n)                         & \text{rewards - costs}
\end{align*}

Actions are given by the system:

\begin{align*}
actions = \{&pickup(t),   & \text{if } ready(t, from) \wedge currentPosition(from) \\
            &move(c),     & \text{if } \lnot currentPosition(c)                    \\
            &deliver(t)\} & \text{if } loaded(t, to) \wedge currentPosition(to)
\end{align*}

The initial state has a given position and a set of tasks ready to be loaded,
an initial position (\texttt{City}), a loading capacity and the money made so
far (which is $0$ at the beginning)

\begin{align*}
    initial = &position(c)                            & c \text{ is the starting city of the agent} \\
              &capacity(n) \; | \; n \in \mathbb{N} \wedge n > 0   & \text{positive capacity} \\
              &ready(t) \; \forall t \in tasks                   & tasks \text{ is the initial \texttt{TaskSet}} \\
              &balance(0) & \text{initial balance is zero}
\end{align*}

The final state only cares that there are no more tasks ready neither loaded.

\begin{align*}
    goal = &position(c) & c \text{ any city} \\
           &capacity(n) & n \text{ the initial capacity} \\
           &\nexists \: ready(t) \; \forall t \in tasks & \text{no more ready tasks} \\
           &\nexists \: loaded(t) \; \forall t \in tasks & \text{no tasks are still loaded} \\
\end{align*}

\medskip
\textbf{Heuristic}

\emph{TODO}

\medskip
\textbf{Performance}

% with the distance heuristic.
\medskip
\begin{tabular}{ r | c c | c c }
    ~ & \multicolumn{2}{c |}{Breadth First Search} & \multicolumn{2}{c}{A* Search} \\
    \cline{2-5}
    \# tasks & \# states & time [s] & \# states & time [s] \\
    \hline
    $6$        & $5e^5$       & $0.9$      & $1e^5$       & $0.3$ \\
    $7$        & $9e^6$       & $17.4$     & $2e^6$       & $7.1$ \\
    $8$        & GC overhead & --     & timeout   & -- \\
\end{tabular}


\end{document}

