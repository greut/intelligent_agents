%!TEX TS-program = xelatex
%!TEX encoding = UTF-8 Unicode
\documentclass[11pt,a4paper]{article}

%\usepackage[left=70pt,top=50pt,bottom=70pt,right=40pt]{geometry}
\usepackage{amsmath}
\usepackage{amsfonts}
\usepackage{amssymb}
\usepackage{fixltx2e}
\usepackage{cmap}
\usepackage{enumerate}
\usepackage{ifthen}
\usepackage{listings}
\usepackage{url}
\usepackage[T1]{fontenc}
%\usepackage{fontspec}
%\usepackage{xunicode}
%\usepackage{xltxtra}
%\setmainfont[Mapping=tex-text,Ligatures={Common,Rare,Discretionary}]{Linux Libertine O}
\usepackage{pdflscape}
\usepackage{alltt}
%\usepackage{algpseudocode}
%\usepackage{wrapfig}
%\usepackage{graphicx}

\ifthenelse{\isundefined{\hypersetup}}{
    \usepackage[colorlinks=true,linkcolor=blue,urlcolor=blue]{hyperref}
    \urlstyle{same}
}{}

\hypersetup{
    pdftitle={Intelligent Agents - EX2 - Yoan Blanc, Tiziano Signo}
}
\title{\phantomsection%
    A Deliberative Agent for the Pickup and Delivery Problem
}
\author{
    Yoan Blanc \texttt{<yoan.blanc@epfl.ch>}, 213552\\
    Tiziano Signo \texttt{<tiziano.signo@epfl.ch>}, 226511
}
\date{\today}


\begin{document}
\maketitle

\noindent
\begin{quote}{\it

    In this exercise, you will learn to use a deliberative agent to solve the
Pickup and Delivery Problem. A deliberative agent does not simply react to
percepts coming from the environment. It can build a plan that specifies the
sequence of actions to be taken in order to reach a certain goal. A
deliberative agent has goals (e.g. to deliver all tasks) and is fully aware of
the world it is acting in.

    Unlike the reactive agent, the deliberative agent knows the list of tasks
that must be delivered. The deliberative agent can therefore construct a plan
(a certain path through the network) that guarantees the optimal delivery of
tasks.

    \begin{enumerate}
        \item Choose a representation for the states, transitions and goals
        (final states) to be used in a state-based search algorithm that finds
        the optimal plan for delivering a set of tasks.

        \item Implement the state-based breadth-first search and A* heuristic
        search algorithms.  Choose one heuristic and explain why. Discuss the
        optimality of your new algorithm in relation to your heuristic.

        \item Implement a deliberative agent which can use the above planning
        algorithms.

        \item Compare the performances of the breadth-first search and the A*
        search algorithms for different problem sizes.

        \item Run the simulation with 1, 2 and 3 deliberative agents and report
        the differences of the joint performance of the agents.

    \end{enumerate}

}\end{quote}

\newpage
\medskip
\textbf{State representation}

The objects of the system are the tasks, cities.

$$ objects = \{T_1, \cdots, T_n\} \cup \{city_1, \cdots, city_m\} $$

Five predicates are kept (for now), one stating where the agent is and 

\begin{align*}
predicates = &city(c)                            & c \text{is a city} \\
             &task(t)                            & t \text{is a task} \\
             &currentPosition(c)                 & \text{position of the agent}      \\
             &ready(t, from, to, weight, reward) & \text{task ready to be picked-up} \\
             &loaded(t, to, weight, reward)      & \text{task currently in transit}  \\
             &delivered(t)                       & \text{task delivered}             \\
             &balance(n)                         & \text{rewards - costs}            
\end{align*}

Actions are given by the system:

\begin{align*}
actions = \{&pickup(t),   & \text{if } ready(t, from) \wedge currentPosition(from) \\
            &move(c),     & \text{if } \lnot currentPosition(c)                    \\
            &deliver(t)\} & \text{if } loaded(t, to) \wedge currentPosition(to)
\end{align*}


\end{document}

