%!TEX TS-program = xelatex
%!TEX encoding = UTF-8 Unicode
\documentclass[11pt,a4paper]{article}

\usepackage[left=120pt,top=80pt,bottom=110pt,right=80pt]{geometry}
\usepackage{amsmath}
\usepackage{amsfonts}
\usepackage{amssymb}
\usepackage{fixltx2e}
\usepackage{cmap}
\usepackage{enumerate}
\usepackage{ifthen}
\usepackage{listings}
\usepackage{url}
\usepackage[T1]{fontenc}
%\usepackage{fontspec}
%\usepackage{xunicode}
%\usepackage{xltxtra}
%\setmainfont[Mapping=tex-text,Ligatures={Common,Rare,Discretionary}]{Linux Libertine O}
\usepackage{pdflscape}
\usepackage{alltt}
\usepackage{algpseudocode}
\usepackage{subfigure}
\usepackage{graphicx}
\usepackage{verbatim}

\ifthenelse{\isundefined{\hypersetup}}{
    \usepackage[colorlinks=true,linkcolor=blue,urlcolor=blue]{hyperref}
    \urlstyle{same}
}{}

\hypersetup{
    pdftitle={Intelligent Agents - EX4 - Yoan Blanc, Tiziano Signo}
}
\title{\phantomsection%
    \Large{An Auctioning Agent for the Pickup \& Delivery Problem}
}
\author{
\textbf{    \small{Group 16}}\\
    \small{Yoan Blanc \texttt{<yoan.blanc@epfl.ch>}, 213552}\\
    \small{Tiziano Signo \texttt{<tiziano.signo@epfl.ch>}, 226511}
}
\date{\small{\today}}


\begin{document}
\maketitle

\noindent
{\it

    You will define an auction strategy for your agent and compete against other agents.
    When designing your strategy, please keep in mind the following facts:

    \begin{itemize}
        \item The minimum price you are willing to accept for delivering the
            task $T_1$ is equal to the marginal cost of delivering $T_1$. In other
            words, you have to:

            \begin{enumerate}
                \item find the cost of the plan for delivering the tasks you've
                    already won,

                \item estimate the cost of the plan for delivering the same
                    tasks plus $T_1$,

                \item and compute the marginal cost of delivering $T_1$ as the
                    difference between the two.

            \end{enumerate}
        \item Any price paid by the auction house which is superior to the
            marginal cost of a task will result in a profit for you to deliver
            that task. Therefore, you should usually bid above your own
            marginal cost. You may occasionally bid below your marginal cost if
            you believe this will reduce the cost of future tasks, but be
            careful not to end up with a deficit!

        \item You do not know what other tasks will be auctioned by the auction
            house. In fact, you do not even know \textbf{how many} more tasks will be
            auctioned after the current one. When you compute your current bid,
            you might want to speculate on the effect of taking the present
            task on the future auctions.

        \item You do know the probability distribution of tasks. The tasks that
            will be auctioned will follow this distribution, so you might want
            to use these probabilities in order to speculate about future tasks
            that will be auctioned.

        \item The winner and all bids are published immediately after each
            auction. This gives you the possibility to refine your strategy
            depending on your opponent(s).

    \end{itemize}
}

\newpage

\section*{Implementation}

We observed that the tasks are in general good to get because of their high
\emph{complementarity}. The general strategy was to grab as many tasks as
possible by trying to predict the total cost of having to deliver more than
$15$ tasks. I won't work for few tasks, sadly. Then, we explored a learning
approach by observing the others' bid, which is surprisingly efficient with
regards to its simplicity.

The planning has been done using the centralized agent work with very few
modifications to deal with a time constraint as well as optimizing on the
newly added tasks directly. before continuing with the \emph{simulated
annealing} hill climbing algorithm.


\subsection*{Agent}

Remark, all our agents are named after Duck Tales characters in English,
French or Italian. Funny codenames to hide their strategies. And they all add
a tax of $1$ to their bid, for fun (and profit). Because you should still be
able to win by asking $1$ more than your calculation.

\subsection*{Dumb agents}

The not so dumb agents, there is no needs to explain the very dumb ones.

\subsubsection*{Picsou}

After doing a pure \emph{greedy} one bidding $1$ about its marginal cost, this
one aims to gain a bit more money by doing the focusing on evicting the other
to take any tasks. It will bid $-1$ below the cost from pickup to delivery
using it's most lightweight vehicle.

It'll win versus very basic agents, relying on the marginal cost only because
it forces you to go below zero if you want to take any task from it.

It relies on overlapping tasks to make profit.

\subsubsection*{Paperino}

This one has a pure price prediction strategy. It computes for each segment of
the map the expected number of times it will appear for a reasonable amount
of rounds ($7$ or $8$ showed to work well). Then it pre-computes based on the
vehicles what cost of one usage based on the total amount it expects to obtain.

It's a pretty agressive strategy that can do well on some setup. It also aims
to grab everything and will fail if the other players takes only the good ones
and leaves it the costly paths. Because it has no knowledge of the other player,
it won't be able to react to its situation either good or bad.

\subsubsection*{Huey}

Based on \emph{Paperino}, it weights some segments differently. If a segment
leads to a city with $[1,2]$ neighbors, then it sees it has more distant. It's
the assumption that whenever you go there, you'll have to come back to continue
your deliveries. So this task's price has to take into account the burden its
causing to the vehicle.

This one is quite good at predicting price but it does not consider the other
player because its goal is to take the good tasks, and disregards the less good
ones.

\subsubsection*{Louie}

This the first historical agent that bids by learning. It takes the bid of the
other agents and computes an average cost per kilometer for each segment.

It's a surprisingly efficient one against the previous agents which have a
cost per segment strategy.

\subsection*{Candidate}
\subsubsection*{Dewey}

This is a mix of \emph{Huey} and \emph{Dewey}, which is using two plannings
to guess the other's marginal cost and overall rewards. It's taking the same
vehicles as itself assuming that it doesn't matter much after a while.

It goes with \emph{Huey} way of working for the first couple of rounds and then
depending on its state it acts differently. If it suspects the other player to
win, then it will switch to \emph{Louie}'s strategy to mimic its opponent. In
case, it thinks he wins, then he will increase its bids with a ratio of the
difference between its estimation and the other's.

\subsection*{Efficiency}

It's very hard to have an always good agent when the tasks can be very
complementary with each other. And learning from the environment is limited by
the assumption that the other agent is doing good. Try to combine too many
variables from the environment makes the bid computation a guestimation rather
then something arguable.


\section*{Conclusion}

We created a third template to get another environment but when running some
tournaments, none of our agents won them all. Knowing how to be properly
aggressive with the starting prices, how to more accurately predict the final
price or when to start acting with respect to your opponents actions are very
hard.

We cannot be confident that any of our agents can bet all the other's one
because they are no clear winners in our experiments either.

One major weakness of our agents is that they are going negative and can stay
there, loosing to a no tasks agent. This agressive behaviour seems weak in some
situations. Not being able to know what's coming next makes the bidding task
hard.

\end{document}
