%!TEX TS-program = xelatex
%!TEX encoding = UTF-8 Unicode
\documentclass[11pt,a4paper]{article}

\usepackage[left=80pt,top=60pt,bottom=110pt,right=60pt]{geometry}
\usepackage{amsmath}
\usepackage{amsfonts}
\usepackage{amssymb}
\usepackage{fixltx2e}
\usepackage{cmap}
\usepackage{enumerate}
\usepackage{ifthen}
\usepackage{listings}
\usepackage{url}
\usepackage[T1]{fontenc}
%\usepackage{fontspec}
%\usepackage{xunicode}
%\usepackage{xltxtra}
%\setmainfont[Mapping=tex-text,Ligatures={Common,Rare,Discretionary}]{Linux Libertine O}
\usepackage{pdflscape}
\usepackage{alltt}
\usepackage{algpseudocode}
\usepackage{subfigure}
\usepackage{graphicx}
\usepackage{verbatim}

\ifthenelse{\isundefined{\hypersetup}}{
    \usepackage[colorlinks=true,linkcolor=blue,urlcolor=blue]{hyperref}
    \urlstyle{same}
}{}

\hypersetup{
    pdftitle={Intelligent Agents - EX4 - Yoan Blanc, Tiziano Signo}
}
\title{\phantomsection%
    An Auctioning Agent for the Pickup and Delivery Problem
}
\author{
    \textbf{Group 16}\\
    Yoan Blanc \texttt{<yoan.blanc@epfl.ch>}, 213552\\
    Tiziano Signo \texttt{<tiziano.signo@epfl.ch>}, 226511
}
\date{\today}


\begin{document}
\maketitle

\noindent
\begin{quote}{\it

    You will define an auction strategy for your agent and compete against other agents.
    When designing your strategy, please keep in mind the following facts:

    \begin{itemize}
        \item The minimum price you are willing to accept for delivering the
            task $T_1$ is equal to the marginal cost of delivering $T_1$. In other
            words, you have to:

            \begin{enumerate}
                \item find the cost of the plan for delivering the tasks you've
                    already won,

                \item estimate the cost of the plan for delivering the same
                    tasks plus $T_1$,

                \item and compute the marginal cost of delivering $T_1$ as the
                    difference between the two.

            \end{enumerate}
        \item Any price paid by the auction house which is superior to the
            marginal cost of a task will result in a profit for you to deliver
            that task. Therefore, you should usually bid above your own
            marginal cost. You may occasionally bid below your marginal cost if
            you believe this will reduce the cost of future tasks, but be
            careful not to end up with a deficit!

        \item You do not know what other tasks will be auctioned by the auction
            house. In fact, you do not even know \textbf{how many} more tasks will be
            auctioned after the current one. When you compute your current bid,
            you might want to speculate on the effect of taking the present
            task on the future auctions.

        \item You do know the probability distribution of tasks. The tasks that
            will be auctioned will follow this distribution, so you might want
            to use these probabilities in order to speculate about future tasks
            that will be auctioned.

        \item The winner and all bids are published immediately after each
            auction. This gives you the possibility to refine your strategy
            depending on your opponent(s).

    \end{itemize}
}\end{quote}

\newpage

\textbf{TODO}
\end{document}
