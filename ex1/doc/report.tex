%!TEX TS-program = xelatex
%!TEX encoding = UTF-8 Unicode
\documentclass[11pt,a4paper]{article}

%\usepackage[left=70pt,top=50pt,bottom=70pt,right=40pt]{geometry}
\usepackage{amsmath}
\usepackage{amsfonts}
\usepackage{amssymb}
\usepackage{fixltx2e}
\usepackage{cmap}
\usepackage{enumerate}
\usepackage{ifthen}
\usepackage{listings}
\usepackage{url}
\usepackage[T1]{fontenc}
%\usepackage{fontspec}
%\usepackage{xunicode}
%\usepackage{xltxtra}
%\setmainfont[Mapping=tex-text,Ligatures={Common,Rare,Discretionary}]{Linux Libertine O}
\usepackage{pdflscape}
\usepackage{alltt}
\usepackage{algpseudocode}

\ifthenelse{\isundefined{\hypersetup}}{
    \usepackage[colorlinks=true,linkcolor=blue,urlcolor=blue]{hyperref}
    \urlstyle{same}
}{}

\hypersetup{
    pdftitle={Intelligent Agents - EX1 - Yoan Blanc, Tiziano Signo}
}
\title{\phantomsection%
    A Reactive Agent for the Pickup and Delivery Problem
}
\author{
    Yoan Blanc \texttt{<yoan.blanc@epfl.ch>}, 213552\\
    Tiziano Signo \texttt{<tiziano.signo@epfl.ch>}, 226511
}
\date{\today}


\begin{document}
\maketitle

\noindent
\begin{quote}{\it

    At the beginning there are two tables:

    \begin{itemize}
        \item $p(i,j)$: The probability that in $\text{city}_i$ there is a task to be transported to $\text{city}_j$.
        \item $r(i,j)$: The reward for a task that is transported from $\text{city}_i$ to $\text{city}_j$.
    \end{itemize}

    \begin{enumerate}
        \item Define your state representation $S$, the possible actions $A$, the
            reward table $R(s,a) | s \in S, a \in A$ and the probability transition
            table $T(s,a,s') | s, s' in S, a \in A$.

        \item Implement the offline reinforcement learning algorithm for
            determining the actions to take in order to search and deliver
            tasks optimally. This algorithm should be executed before the
            vehicles start moving.

        \item Run simulations of one, two and three agents using your optimally
            learned strategy $V(S)$. Look at the performance graph of the agents.
            How does it change for different discount factor $\gamma$?

    \end{enumerate}
}\end{quote}

\newpage
\medskip
\textbf{State representation}

The space of states $S$ is defined by tuples of the current city ($curr$)
and the destination city ($dest$). We made this call because that what
is given to the agent as informations.

$$ \text{cities} = \{ \text{city}_1, \text{city}_2, \text{city}_3, \ldots, \text{city}_n \}$$

$$ S = \{ (curr, dest) \quad | \quad curr \in \text{cities}, dest \in
\text{cities} \cup \{ \emptyset \}, curr \neq dest \} $$

For example: $(city_i, city_j)$ represents the state where the agent is
in $city_i$ and there is a delivery to $city_j$ available to it. Likewise,
$(city_k, \emptyset)$ means that there are not tasks available to the agent
while it travels through $city_k$.

\medskip
\textbf{Action representation}

The space of actions $A$ is defined by accepting and delivering the proposed
task ($deliver$) or by deciding to move to another city.

$$ A = \{ action \quad | \quad action \in \text{cities} \cup \{\text{deliver}\} \} $$

\medskip
\textbf{Other definitions}

With the two spaces, we can now define the reward function $R(s,a)$ as well as
the state transition function $T(s,a,s')$ using $p(i,j)$ and $r(i,j)$. First,
define another function $q(i,j)$ which represents the cost to travel from
the $city_i$ to the $city_j$.

$$ q(i, j) = \text{cost per kilometer} * distance(city_i, city_j) $$

The reward function is defined as follow:

$$ R(s, a) = \left\{
    \begin{array}{l l}
        r(s.curr, s.dest) - q(s.curr, s.dest) & \text{if} \quad a \in \text{cities} \\
        q(s.curr, s.dest) & \text{if} \quad a = \text{deliver}
    \end{array} \right. $$

And the state transistion function:

% shall we only consider the neighbors city when not delivering packages?

$$ T(s, a, s') = \left\{
    \begin{array}{l l}
        p(s'.curr, s'.dest) & \text{if} \quad a = s'.curr \\
        p(s'.curr, s'.dest) & \text{if} \quad a = \text{deliver}, s.dest = s'.curr \\
        0 & \text{otherwise}
    \end{array} \right. $$

\medskip
\textbf{Algorithm}

We followed the algorithm from the slides. An extra variable $ Best$ is used to
store which $a$ was the $max$ one and not only the best value.

\begin{algorithmic}
    \Function{reinforcementLearning}{}
        \For{$s \in S$}
            \State $V(s) \gets 0$
        \EndFor
        \While{$error \leq \epsilon$}
            \For{$s \in S$}
                \For{$a \in A$}
                    $Q(s, a) \gets R(s, a) + \gamma \sum_{s' \in S} T(s,a,s') V(s')$
                \EndFor
            $V(s) \gets max_aQ(s, a)$
            \EndFor
        \EndWhile
    \EndFunction
\end{algorithmic}

\bigskip
\textbf{Implementation details}

We start defining specifically built classes for \texttt{State} and
\texttt{Action} in order to simplify the construction of reward and transition
matrices and the reinforcemente learning algorithm computation.

The implementation of the agents is based on 2 functions: \texttt{setup} and
\texttt{act}, the first one handling the offline preprocessing, and the second
one used at runtime to decide what to do and where to go next from any city.

The function \texttt{setup} starts with the creation of our 4 basic structures
required for the reinforcemente learning algorithm:

\begin{enumerate}
    \item{\texttt{State} set (as an array of objects $s$)}
    \item{\texttt{Action} set (as an array of objects $a$)}
    \item{\texttt{Reward} matrix (as a matrix $s \times a$)}
    \item{\texttt{Transition} matrix (as a matrix $s \times a \times s'$)}
\end{enumerate}

Each one of these objects is built using specific separated functions,
respectively we have \texttt{buildStateList}, \texttt{buildActionList},
\texttt{computeRewards} and \texttt{computeTransitions}.

The content of these functions is summarized and explained previously above,
but one important note has to be given: since the reward matrix considers
all the possible combinations $(s, a)$ and our problem have unallowed or
impossible cases (e.g. delivery when we have no task) it's necessarnot to
consider them. In this unallowed cases we specify the reward as a very large
negative value (\texttt{Double.NEGATIVE\_INFINITY}).

Thanks to how we handle these cases the reinforcement learning algorithm
(called right after in the \texttt{setup} function) simply follows the
directives given in class and is free of additional verifications. As an output
we obtain all the best actions to take from any given state.

The final part IS YOAN WHO EXPLAINS THE GENERATED GRAPH.\newline

When everything is computed the simulation can be started and here the focus is
moved to the \texttt{act} function. This simple function only cares about
applying the best action to take from the given state.


\bigskip
\textbf{Agents comparison}

In our simulation, we show the map given in the configuration and 3 agents
operating on it. Each one of the agents follows a different algorithm:

\begin{itemize}
    \item{random choice (random choice to accept a task, move to random neighbor if not accepted)}
    \item{greedy choice (always accept a task, move to random neighbor otherwise)}
    \item{reactive choice (take the best action)}
\end{itemize}

While the simulation runs it's possible to see the differences of rewards over
time and compare the efficiency of each agent.

\end{document}
