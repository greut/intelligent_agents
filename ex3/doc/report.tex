%!TEX TS-program = xelatex
%!TEX encoding = UTF-8 Unicode
\documentclass[11pt,a4paper]{article}

%\usepackage[left=70pt,top=50pt,bottom=70pt,right=40pt]{geometry}
\usepackage{amsmath}
\usepackage{amsfonts}
\usepackage{amssymb}
\usepackage{fixltx2e}
\usepackage{cmap}
\usepackage{enumerate}
\usepackage{ifthen}
\usepackage{listings}
\usepackage{url}
\usepackage[T1]{fontenc}
%\usepackage{fontspec}
%\usepackage{xunicode}
%\usepackage{xltxtra}
%\setmainfont[Mapping=tex-text,Ligatures={Common,Rare,Discretionary}]{Linux Libertine O}
\usepackage{pdflscape}
\usepackage{alltt}
%\usepackage{algpseudocode}
%\usepackage{wrapfig}
%\usepackage{graphicx}

\ifthenelse{\isundefined{\hypersetup}}{
    \usepackage[colorlinks=true,linkcolor=blue,urlcolor=blue]{hyperref}
    \urlstyle{same}
}{}

\hypersetup{
    pdftitle={Intelligent Agents - EX3 - Yoan Blanc, Tiziano Signo}
}
\title{\phantomsection%
    A Centralized Agent for the Pickup and Delivery Problem
}
\author{
    \textbf{Group 16}\\
    Yoan Blanc \texttt{<yoan.blanc@epfl.ch>}, 213552\\
    Tiziano Signo \texttt{<tiziano.signo@epfl.ch>}, 226511
}
\date{\today}


\begin{document}
\maketitle

\noindent
\begin{quote}{\it

    In this exercise you will implement the Stochastic Local Search algorithm
    (SLS) to find an efficient solution to the CSP description of the PDP.

    \begin{enumerate}
        \item Implement the Stochastic Local Search algorithm for the PDP.

        \item Run simulations for different configurations of the environment
        (i.e. different tasks and number of vehicles) in order to observe the
        behavior of the centralized planner using the SLS algorithm.

        \item Reflect on the fairness of the optimal plans. Observe that
        optimality requires some vehicles to do more work than others.
        Illustrate this observation with an example in your report.

        \item Test your program for different number of vehicles and various
        sizes of the task set. How does the complexity of your algorithm depend
        on these numbers?

    \end{enumerate}

}\end{quote}

\newpage
\subsection*{Food for Thoughts}

Very tasks $t_i$ is split into two actions: pickup and delivery.

\begin{align*}
actions = \{&t_1^p,                                     & \text{task $1$ pickup} \\
          &t_1^d,                                     & \text{task $1$ delivery} \\
          &t_2^p, &_2^d,                              & \\
          &\dots,                                     & \\
          &t_{N_T}^p, t_{N_T}^d\}                     & \text{up to the end}
\end{align*}

The set of plans is formed of tuples, one per vehicle, and the corresponding
list of action to do. This is as strong constraints:

\begin{enumerate}
    \item two actions belonging to a same task must be in the same action,

    \item pickup and delivery must appear in that order,

    \item each item appear once and only once among the plans.
\end{enumerate}

\begin{align*}
plans = \{&(v_1, (t_j^p, t_j^d, \: \dots)),            & \text{vehicle $1$ and task $j$} \\
          &(v_2, (t_k^p, t_l^p, \dots, t_k^d, t_l^d)), & \text{interleaved tasks} \\
          &\dots,                                      & \\
          &(v_{N_V}, \varnothing)\}                    & \text{actions can be empty too}
\end{align*}

\subsubsection*{Local operators}

A task can be affected by two kinds of changes: its order within a vehicle or
the vehicle it belongs to.

\begin{enumerate}
    \item \textbf{Task order within a vehicle} \\
        Moving only one action, either pickup or deliver, it will try to
        move them to any other acceptable solution and pick the best one.

        pickup can be moved ahead up to the first state or as long as the
        vehicle capacity is not reached. And moved forward up to the
        corresponding delivery.

        delivery can be moved ahead up to the pickup action and forward
        up to the end of the list as long as the maximum capacity is not
        already reached.

    \item \textbf{Vehicle doing the task} \\
        To move a task to another vehicle, it has to pick both pickup and
        delivery and try them on another vehicle.

        For each vehicle, it'll try to find the position in the list that
        minimize the overall cost.

        % Try to find the optimal solution for both p and d is way too
        % costly.
\end{enumerate}


\end{document}

